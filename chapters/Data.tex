% !TEX root = ../main.tex

\chapter{Data Preparation}

In this chapter, we describe what data was used to train and evaluate the inference system.
We discuss the parameters used to define the Topology Optimization problem and their variability and properties.
The other aspect to be covered is the ways to represent these data in a way suitable for ANN system input.
Lastly, we discuss the aspects of choosing the proper training dataset for the inference system.
\medskip

\section{Topology Optimization Problem Cases}

The physical problem, for which the Topology Optimization (TO) is performed, is defined by the Boundary Conditions (BC) and the force applied to the element. 
The former has a view of Dirichlet BC and defines the parts of the element that experience no displacement and the latter is in general form a filed of forces applied to all the point of the element. 
Translated in the discretized form used in the Finite Element Analysis (FEA) performed during the TO, the former defined the fixed degrees of freedom and the latter defines the non-zero elements of the right-hand-side vector of loads.
\medskip

To obtain a set of TO results that could be used as the training and testing samples for the Artificial Neural Network (ANN) we choose to perform TO for various combination of values of the boundary conditions and load cases.
\medskip

\subsection{2D Cases of Topology Optimization}

As Boundary Condition we can choose to fix one of the sides of the domain (one of four walls). 
We can also choose to leave the inner $\frac{1}{3}$ of the side unfixed or fix only the extreme points of the wall. 
\begin{figure}
	\centering
	%\begin{multicols}{3}
	\subcaptionbox{Solid wall}[.3\linewidth][c]{
		\includegraphics[width=0.3\linewidth]{images/TOcase_init.pdf}}
	\subcaptionbox{A gap in the wall}[.3\linewidth][c]{
		\includegraphics[width=0.28\linewidth]{images/TOcase_23bc.pdf}}
	\subcaptionbox{Corner points fixed}[.3\linewidth][c]{
		\includegraphics[width=0.29\linewidth]{images/TOcase_2pbcsvg.pdf}}
	%\end{multicols}
	\caption{Boundary Conditions considered for 2D TO cases }
	\label{fig:3bcs}
\end{figure}
\medskip

As a simple load case, we consider only the situation where a force is applied to a single point of the element. 
The variable parameters are the coordinates of the force application $(X, Y)$ and the direction of the force.
We keep the magnitude of the force in every direction constant and choose one of the eight possible directions: along one of the axes in one of the direction or a combination of those.
The last parameter that defines the result of TO is the value of volume fraction occupied by the material, for which we choose $3$ values: $0.1, 0.2, 0.4$
\medskip

In the end, we obtain the number of possible input sample equal to $ 4\,\mathrm{(fixed \, side)} \; \times \; 3\,\mathrm{(type \, of \, BC)} \\ \; \times \; 3\,\mathrm{(value \; of \; Volume \; fraction)} \; \times \; 8\,\mathrm{(force \, direction)} \; \times \; 32\,\mathrm{(Y \, coordinate)} \; \times \; 16\,\mathrm{(X \, coordinate)} = 147446 $ bitmaps.
\medskip

Because of the symmetry of the square problem domain, or every case for a single wall fixed as a boundary condition, there are cases equivalent up to a rotation on $ 30^\circ, 180^\circ, 270^\circ $ angles.
That means it sufficient to generate layout only for a single type edge of domain fixed and then apply \emph{data augmentation} to obtain all the other results.
It is a simple transformation applied to every sample of the training dataset, a rotation in our case, that allows having a richer set with easier means.
In our case, we reduced time of dataset generation by almost $4$ time since we had 4 possible edges to fix.
\medskip

\subsection{3D Cases of Topology Optimization}

For the second series of experiments, we required 3D models of optimized elements.
Because of the larger design space, and larger problem size, which lead to much larger time spent to obtain a single result, we decided to restrict data variability.
For the 3D case, we left only a single volume fraction value equal to $0.2$, as well as single Boundary Condition case when there are four corner elements of the $ X=0 $ edge of the domain are fixed.
The problem domain we considered has the size of $32 \times 32 \times 32$ hexahedral elements, and we performed TO for cases when a force was applied to a single element. 
We generated TO results not for every possible point of load application but took point with the step of 4 in every dimension in the furthermost half of the domain in $Y$ direction. 
For every point of load application, we considered forces of magnitude $1$ directed to each face of the hexahedral element, as well as towards each vertex, which corresponds to $D3Q15$ model of hexahedral directions. \todo{maybe unsuitable}
In total, for the entire dataset, we generated $ 14\,\mathrm{(force \, direction)} \; \times \; 5\,\mathrm{(Y \, coordinate)} \; \times \; 9\,\mathrm{(X \, coordinate)} \times \; 9\,\mathrm{(Y \, coordinate)} = 5670 $ models.
\medskip

\section{Dataset generation}

For the purpose of generating the dataset of known solved TO problems, in this work we used the MATLAB code provided in the paper "A 99 line topology optimization code written in Matlab"\cite{to_99line}.
It minimizes compliance for 2D material layout under some load.
The code uses square elements for finite element discretization and uses the power-law model to represent the compliance as the goal function, as described in the introduction section.
\medskip

The code was modified in order to have several arbitrary points where the load is applied.
A solution of single TO problem is called by calling the function, where all are passed as parameters, such problem dimension, volume fraction, power penalty, convergence criterion, coordinated of force application points and force vector. 
The script saved the resulting material layout as a gray-scale PNG image where every pixel value denotes the virtual material density of corresponding square finite element.
For the typical problem size of $ 32 \times 32 $ single TO problem solution took averagely $1 \mathrm{s}$ using a single CPU.
Totally there were $~150$ thousand created.
\todo{fill in reference, number, code etc.} 
\medskip

To generate results of compliance minimization for 3D models we used a program IDeAs\cite{}, created a Siemens CT.
The program the parameters of the TO problem are defined in a JSON\cite{bibl:json} file, in which one should state STEP\cite{bibl:step} files describing the geometrical domain, fixed regions for boundary conditions, and surfaces at which force is applied.
The boundary condition type, force vector, and volume fraction value are also stated, as well as parameters of physics, discretization, and analyzes. 
\medskip

Run with $32$ threads on a single Nvidia GeForce 1080 GTX GPU we generated $~5$ thousands of $ 32 \times 32 \times 32$ voxels model of result models, which took $10$ seconds.
\medskip 

% The Boundary Coditions used during Topology Optimization as an input to the Finite Element Analysis constitute of the fixed degrees of freedom


\section{Inputs of the ANN Model}

In our case for every fixed domain, we use every TO problem, is defined by boundary conditions, filed of applied forces and volume fraction occupied by material. 
These are three parameters that vary from case to case of TO problem and others stay the same, thus the model should be able to infer proper layout using this varying information.
One of the issues during the design of the model was how the model should account for this information.
\medskip

The boundary condition is defined by a geometrical locus on the domain for which values of some function is predefined, in our case it is the displacement vector for these points is equal to zero. 
After the discretization, it is expressed in having some degrees of freedom fixed. 
%Essentially that means that for some of the geometrical elements the component in the displacement vector should be always put to zero which is reflected in the form of the right-hand side 
Since our domain has a correspondence with a bitmap where every element of the TO problem maps to an element of an array, we represent the boundary conditions with an array where elements with indices corresponding to coordinates of the fixed element have a value of $1$ and $0$ for everything else.
\medskip

In a similar manner, we represent the force field applied to the element.
%Generally we have only a few points at which forces are applied, which after discretization will be reflected in a few elements for which there will be values filled in the right-hand side.
We have two different bitmaps for $X$ and $Y$ components of the force or every point at which force is applied we fill the array element with the $+1$ or $-1$ value depending on the force direction, leaving everything else $0$.
\medskip

The Volume Fraction is essentially a scalar value and it is accounted in the system by introducing a new channel tiled with a single value at the layer of the representation in the lowest dimensionality as described in the ``Architecture'' section\ref{sec:arch}.
\medskip

The bitmap of boundaries, force in $X$ direction and force in $Y$ direction form the three input channels of the model on which the convolutions are applied.
\medskip

The one problem with such representation of BC and force fields is that there are only a few non-zero elements in the channel, that leads very little information influx in the model and makes it virtually impossible to train the model.
\todo{explain more, demonstrate?}
\medskip

As a remedy for this problem we populate the input channels with more non-zero elements representing meaningful information by finding a \emph{Signed Distance Field (SDF)} of every input channel\cite{bibl:sdf}.
\medskip

For every channel, we consider the non-zero elements as important and put a virtual surface around them in order to find signed distance to them for every element of the bitmap.
The ambiguity for the force direction is solved by defining inner part of zero-surface as either inner or outer part, depending on the direction of the force, while computing the SDF.
\medskip

\subsection{Signed Distance Field}

One of the ways to describe a set of geometrical points $\Omega$ within the domain is defining the distance to the boundary of $\Omega$ for every point of the domain. 
The values of such function are typically negative if the point is located within $\Omega$ and positive is outside.
In the end, we define the Signed Distance Field as a metric space $X$ with metrics $f$, or \emph{signed distance function} :    
\begin{equation}
f(x) = \begin{cases}
			d(x, \partial \Omega) & \mbox{ if } x\in\Omega  \\
			-d(x, \partial \Omega) & \mbox{ if } x\in\Omega^c 
\end{cases} 
	, \quad \mathrm{where} \; d(x, \partial \Omega) := \inf_{y \in \partial \Omega}d(x, y)
\end{equation} 

If $\Omega$ is a subset of Euclidean space and has smooth boundaries, then the gradient of the metrics is constant at every point and satisfies the \emph{Eikonal equation}.
\medskip

\begin{equation}
 |\nabla f(x)| = \mathbf{1}, \quad \forall x \in X
\end{equation}

This equation describes the propagation of the wavefront started from the boundary of the $\Omega$ and its solution will describe travel time of the wavefront to every point in case of the constant speed of propagation which is equal to the signed distance function.
\begin{figure}[ht]
	\centering
	\includegraphics[width=0.5\linewidth]{images/sdf_dem1.png}
	\label{fig:sdf_ex}
	\caption{The example of an SDF. The boundary of the region (curve) is represented by distance to the curve at every point (heatmap).}
\end{figure}
\medskip

There is a fast way to solve this equation for the points located on the regular grid called \emph{Fast Marching Method (FMM)}.
It is a graph traversing algorithm based on the Dijkstra's algorithm \cite{bibl:fmm} and it's complexity is $ O(M \log M) $ where $M$ is the number of grid points.
There are plenty of FMM implementations available, and we chose to use python package \emph{ski-fmm} \cite{scikit-fmm}, a part of SciKit framework that allows obtaining SFD as Numpy arrays\cite{bibl:numpy} by calling only one function passing array describing zero-boundary to it.
\medskip

In this way, we have a fast and easy way to describe our input images having all the pixels populated with meaningful values.
\medskip

\subsection{Training Data Set}

One of the main goals of the work was to estimate the capabilities of such approach to producing the model having high accuracy, trained using as small train dataset as possible.
For 3D case preparing every data sample was taking more than a second for a relatively coarse resolution, which means that preparing large dataset, together with training of the network, will diminish the speedup gained by applying the inference system.  
\todo{get real performance}
\medskip

To define the influence of chosen trained dataset on the resulting model and its performance, we chose several of those with a different number of samples and with diffident distributions of parameters within. 
One of the main questions was also how to chose the samples with respect to the points where loads are applied for the case, so we created data sets each having different patterns for the coordinate of force application point and trained models for those to see which one gives a better model performance.
Among the patterns, we have chosen the next ones: a regular grid, checkered pattern, a sparse grid and force applied at the edges.

\begin{figure*}[ht]
	\centering
\begin{multicols}{3}
	\includegraphics[width=0.8\linewidth]{images/loads_layout_edges.png}\par
	\includegraphics[width=0.8\linewidth]{images/loads_layout_sparse.png}\par
	\includegraphics[width=0.8\linewidth]{images/loads_layout_cross.png}\par
\end{multicols}
\label{fig:sampling_layout}
\caption{Comparison of different sampling patterns for training dataset. Every pixel denotes a sample, coordinates of pixel groups denote a coordinate of force application point, every pixel around denotes a direction of the force. Figures show the images used to diagnose the real data sets used. On the left, the dataset with the domination of forces applied to the edge of the problem domain. The middle one demonstrates the sparse coordinates chosen for the force application. right one demonstrates coordinates taken among main lines.   
}
\end{figure*}
\medskip
 

 